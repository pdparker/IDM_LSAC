\documentclass[man]{apa6}
\usepackage{lmodern}
\usepackage{amssymb,amsmath}
\usepackage{ifxetex,ifluatex}
\usepackage{fixltx2e} % provides \textsubscript
\ifnum 0\ifxetex 1\fi\ifluatex 1\fi=0 % if pdftex
  \usepackage[T1]{fontenc}
  \usepackage[utf8]{inputenc}
\else % if luatex or xelatex
  \ifxetex
    \usepackage{mathspec}
  \else
    \usepackage{fontspec}
  \fi
  \defaultfontfeatures{Ligatures=TeX,Scale=MatchLowercase}
\fi
% use upquote if available, for straight quotes in verbatim environments
\IfFileExists{upquote.sty}{\usepackage{upquote}}{}
% use microtype if available
\IfFileExists{microtype.sty}{%
\usepackage{microtype}
\UseMicrotypeSet[protrusion]{basicmath} % disable protrusion for tt fonts
}{}
\usepackage{hyperref}
\hypersetup{unicode=true,
            pdftitle={The title},
            pdfauthor={First Author~\& Philip Parker},
            pdfkeywords={keywords},
            pdfborder={0 0 0},
            breaklinks=true}
\urlstyle{same}  % don't use monospace font for urls
\usepackage{graphicx,grffile}
\makeatletter
\def\maxwidth{\ifdim\Gin@nat@width>\linewidth\linewidth\else\Gin@nat@width\fi}
\def\maxheight{\ifdim\Gin@nat@height>\textheight\textheight\else\Gin@nat@height\fi}
\makeatother
% Scale images if necessary, so that they will not overflow the page
% margins by default, and it is still possible to overwrite the defaults
% using explicit options in \includegraphics[width, height, ...]{}
\setkeys{Gin}{width=\maxwidth,height=\maxheight,keepaspectratio}
\IfFileExists{parskip.sty}{%
\usepackage{parskip}
}{% else
\setlength{\parindent}{0pt}
\setlength{\parskip}{6pt plus 2pt minus 1pt}
}
\setlength{\emergencystretch}{3em}  % prevent overfull lines
\providecommand{\tightlist}{%
  \setlength{\itemsep}{0pt}\setlength{\parskip}{0pt}}
\setcounter{secnumdepth}{0}
% Redefines (sub)paragraphs to behave more like sections
\ifx\paragraph\undefined\else
\let\oldparagraph\paragraph
\renewcommand{\paragraph}[1]{\oldparagraph{#1}\mbox{}}
\fi
\ifx\subparagraph\undefined\else
\let\oldsubparagraph\subparagraph
\renewcommand{\subparagraph}[1]{\oldsubparagraph{#1}\mbox{}}
\fi

%%% Use protect on footnotes to avoid problems with footnotes in titles
\let\rmarkdownfootnote\footnote%
\def\footnote{\protect\rmarkdownfootnote}


  \title{The title}
    \author{First Author\textsuperscript{1}~\& Philip Parker\textsuperscript{1}}
    \date{}
  
\shorttitle{Title}
\affiliation{
\vspace{0.5cm}
\textsuperscript{1} Institute for Positive Psychology and Education, Australian Catholic University}
\keywords{keywords\newline\indent Word count: X}
\usepackage{csquotes}
\usepackage{upgreek}
\captionsetup{font=singlespacing,justification=justified}

\usepackage{longtable}
\usepackage{lscape}
\usepackage{multirow}
\usepackage{tabularx}
\usepackage[flushleft]{threeparttable}
\usepackage{threeparttablex}

\newenvironment{lltable}{\begin{landscape}\begin{center}\begin{ThreePartTable}}{\end{ThreePartTable}\end{center}\end{landscape}}

\makeatletter
\newcommand\LastLTentrywidth{1em}
\newlength\longtablewidth
\setlength{\longtablewidth}{1in}
\newcommand{\getlongtablewidth}{\begingroup \ifcsname LT@\roman{LT@tables}\endcsname \global\longtablewidth=0pt \renewcommand{\LT@entry}[2]{\global\advance\longtablewidth by ##2\relax\gdef\LastLTentrywidth{##2}}\@nameuse{LT@\roman{LT@tables}} \fi \endgroup}


\DeclareDelayedFloatFlavor{ThreePartTable}{table}
\DeclareDelayedFloatFlavor{lltable}{table}
\DeclareDelayedFloatFlavor*{longtable}{table}
\makeatletter
\renewcommand{\efloat@iwrite}[1]{\immediate\expandafter\protected@write\csname efloat@post#1\endcsname{}}
\makeatother

\authornote{Add complete departmental affiliations for each author here. Each new line herein must be indented, like this line.

Enter author note here.

Correspondence concerning this article should be addressed to First Author, 33 Berry Street North Sydney. E-mail: \href{mailto:philip.parker@acu.edu.au}{\nolinkurl{philip.parker@acu.edu.au}}}

\abstract{
One or two sentences providing a \textbf{basic introduction} to the field, comprehensible to a scientist in any discipline.

Two to three sentences of \textbf{more detailed background}, comprehensible to scientists in related disciplines.

One sentence clearly stating the \textbf{general problem} being addressed by this particular study.

One sentence summarizing the main result (with the words ``\textbf{here we show}'' or their equivalent).

Two or three sentences explaining what the \textbf{main result} reveals in direct comparison to what was thought to be the case previously, or how the main result adds to previous knowledge.

One or two sentences to put the results into a more \textbf{general context}.

Two or three sentences to provide a \textbf{broader perspective}, readily comprehensible to a scientist in any discipline.


}

\begin{document}
\maketitle

\hypertarget{methods}{%
\section{Methods}\label{methods}}

We report how we determined our sample size, all data exclusions (if any), all manipulations, and all measures in the study.

\hypertarget{participants}{%
\subsection{Participants}\label{participants}}

\hypertarget{material}{%
\subsection{Material}\label{material}}

\hypertarget{procedure}{%
\subsection{Procedure}\label{procedure}}

\hypertarget{data-analysis}{%
\subsection{Data analysis}\label{data-analysis}}

\hypertarget{results}{%
\section{Results}\label{results}}

\hypertarget{parents-assessment-of-their-childs-math-and-reading-relative-ability} & \multicolumn{1}{c}{97.5 \%} & \multicolumn{1}{c}{p.value}\\
\midrule
Math Interest & 1.00 & 0.29 & 0.21 & 0.36 & 0.00\\
Math Interest & 2.00 & -0.10 & -0.17 & -0.02 & 0.01\\
Math Interest & 3.00 & 0.07 & -0.02 & 0.16 & 0.15\\
Reading Interest & 1.00 & 0.30 & 0.22 & 0.38 & 0.00\\
Reading Interest & 2.00 & -0.10 & -0.19 & -0.01 & 0.03\\
Reading Interest & 3.00 & 0.05 & -0.04 & 0.15 & 0.28\\
\bottomrule
\end{tabular}
\end{threeparttable}
\end{center}
\end{table}

\hypertarget{math-and-reading-interest} & \multicolumn{1}{c}{97.5 \%} & \multicolumn{1}{c}{p.value}\\
\midrule
Math Interest & 1.00 & 0.12 & 0.04 & 0.20 & 0.00\\
Math Interest & 2.00 & -0.14 & -0.23 & -0.05 & 0.00\\
Math Interest & 3.00 & -0.07 & -0.16 & 0.03 & 0.17\\
Reading Interest & 1.00 & 0.24 & 0.16 & 0.32 & 0.00\\
Reading Interest & 2.00 & 0.05 & -0.04 & 0.14 & 0.29\\
Reading Interest & 3.00 & 0.12 & 0.02 & 0.21 & 0.02\\
\bottomrule
\end{tabular}
\end{threeparttable}
\end{center}
\end{table}

\hypertarget{discussion}{%
\section{Discussion}\label{discussion}}


\end{document}
